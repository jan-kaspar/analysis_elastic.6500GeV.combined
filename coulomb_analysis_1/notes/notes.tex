\input macros

\def\baseDir{/afs/cern.ch/work/j/jkaspar/analyses/elastic/6500GeV/combined/coulomb_analysis_1}

%----------------------------------------------------------------------------------------------------

\hbox{}
\vskip-10mm

\centerline{\SetFontSizesXX Coulomb analysis, $\sqrt s = 13\un{TeV}$}
\vskip2mm
\centerline{version: {\it \number\day. \number\month. \number\year}}

\vfil
\InsertToc

\vfil
\eject

\BeginText

%----------------------------------------------------------------------------------------------------
\chapter[datasets]{Datasets}

\> 2-RP analysis
\>> data
\>>> file: /afs/cern.ch/work/j/jkaspar/analyses/elastic/6500GeV/beta2500/2rp/DS-merged/merged.root
\>>> object: [binning]/merged/combined/h\_dsdt
\>> systematics
\>>> file: /afs/cern.ch/work/j/jkaspar/analyses/elastic/6500GeV/beta2500/2rp/systematics/matrix.root
\>>> object: matrices/all-but-norm/[binning]/cov\_mat


%----------------------------------------------------------------------------------------------------
\chapter[modelling]{Modelling of Coulomb-nuclear interference}

\> hadronic amplitude
\>> modulus
\>>> low $|t|$: exponential form with variable parameters $a$ and $b_n$
\eq{{\cal A}^{\rm N} = a \exp\left(\sum_{n = 1} b_n t^n\right)}
\>>> high $|t|$: fixed amplitude describing well the dip/bump structure, 4 different parametrisations considered: \plot{high_t_part/high_t_part_fit_cmp.pdf}
\>> the fit "fitN-1" doesn't describe the dip well and should be thus handled with caution
\>>> medium $|t|$: smooth transition between the low- and high-$|t|$, implemented via error function
\>> phase
\>>> for the moment only central phases considered, only one variable parameter $\rh$
\>>> default: constant
\>>> tests: Bailly, standard

\> Coulomb amplitude: QED OPE + experimental form factors
\>> form factors: Puckett (default), Arrington and Borkowski for test

\> interference formula: Kundrat-Lokajicek (KL)

\> implementation: Elegent v2


%----------------------------------------------------------------------------------------------------
\chapter[fit]{Fit procedure}

\> input
\>> $\d\si/\d t$ histogram: central values and statistical uncertainties
\>> covariance matrix of relative systematic uncertainties

\> program options
\>> which uncertainties used: stat, syst, norm
\>> $|t|_{\rm max}$: also has impact on where the low $|t|$ parametrisation is used

\> procedure = sequence of several fit steps
\>> number of fit steps
\>>> default: 3
\>>> test: 10

\>> initial point
\>>> default: $a = 1.84E9$, $b_1 = 10.2$, $b_2 = b_3 = 0$, $\rh = 0.12$
\>>> test 1: $a = 1.84E9$, $b_1 = 10.2$, $b_2 = b_3 = 0$, $\rh = 0.06$
\>>> test 2: $a = 1.84E9$, $b_1 = 9.9$, $b_2 = b_3 = 0$, $\rh = 0.12$
\>>> test 3: $a = 1.70E9$, $b_1 = 10.2$, $b_2 = b_3 = 0$, $\rh = 0.12$

\> between fit steps
\>> adjust bin representative points: for each bin find $t^{\rm rep}$ such that
\eq{f(t^{\rm rep}) = {1\over w} \int_{\rm bin} f(t)\,\d t}
where $w$ represents the width of the bin

\>> evaluate the covariance matrix of the systematics
\>>> combine the matrix of relative systematics (input) with the last fit result

\>> sample $\Psi$ function from the KL formula (to be used in the next fit step)

\> each fit step
\>> $\ch^2$ minimisation by ROOT::Minuit2
\>> $\ch^2$ defined
\eq{\ch^2 = \sum_{i,j} \De_i \mat V^{-1}_{ij} \De_j,\quad \De_i = c_i - f(t^{\rm rep}_i)}
where $i,j$ are bin indeces, $\mat V$ is the covariance matrix (depending on which uncertainties are included), $c_i$ refers to the bin content of the $\d\si/\d t$ histogram, $f$ is the fit function and $t^{\rm rep}_i$ is the representative point of a fit

%
%	p0_lim_min = 1.32; p0_lim_max = 1.67;


%----------------------------------------------------------------------------------------------------
\chapter{Validation}

\section[validation-mc]{Validation with MC}

\> method
\>> generate $N$ samples with realistic but identical settings
\>> fit the $N$ samples
\>> do statistics on the $N$ fit results: bias (mean difference fit - truth), uncertainty (fluctuation of fit - truth)

\> realistic simulation
\>> binning as in data
\>> statistical uncertainty taken from data (final merged histogram)
\>> systematics taken from the same matrix as used for fitting
\>> \plot{simulation/uncertainties.pdf}: summary of uncertainties used, as function of $t$
\>>> statistical: red
\>>> systematic: green (as read from input), blue dashed (as obtained from the generator matrix -- control)

\> truth models
\>> exp1 and exp3, with parameters tuned to data
\>> $\rh = 0.10$ and $0.14$
\>> KL formula
\>> \plot{simulation/model_example.pdf}: summary of models considered

\> validation of the simulation procedure
\>> method: generate 400 samples and extract fluctuation of each bin
\>> results: \plot{simulation/validation.pdf}
\>>> bias (green): compatible with zero
\>>> RMS: in general good agreement between input (red) and simulation (blue), the small discrepancy seems to be a feature of ROOT random generator and was checked to decrease with increasing sample size

\> fit bias and uncertainty
\>> $a$ = hadronic cross-section at $t=0$
\>> $\De$ = fit result - truth
\>> \plot{simulation/fit_bias_uncertainty.pdf}: result summary
\>>> bias (mean of $\De$): compatible with 0
\>>> uncertainty (RMS of $\De$): value reasonable, increasing with increasing amount of uncertainties simulated

\> additional study: impact of normalisation error on fit parameters
\>> \plot{simulation/normalisation_error_impact.pdf}: for various truth models shows the fit bias as a function of simulated relative normalisation error









%----------------------------------------------------------------------------------------------------
\chapter[results]{Results}

\section[results-checks]{Reference fits}

\centerline{\vbox{\halign{%
\vrule\strut\ \hfil#\hfil\ &\vrule\ \hfil\ #\ \hfil\vrule&\ \hfil#\ \hfil\vrule\cr\ln
fit type, binning & $t_{\rm max} = 0.07\un{GeV^2}$ & $t_{\rm max} = 0.15\un{GeV^2}$\cr\ln
%
exp1, ob-1-20-0.05 & \plot{reference_fits/ob-1-20-0.05_exp1_0.07.pdf} & \plot{reference_fits/ob-1-20-0.05_exp1_0.15.pdf} \cr
exp1, ob-2-10-0.05 & \plot{reference_fits/ob-2-10-0.05_exp1_0.07.pdf} & \plot{reference_fits/ob-2-10-0.05_exp1_0.15.pdf} \cr
exp1, ob-3-5-0.05  &  \plot{reference_fits/ob-3-5-0.05_exp1_0.07.pdf} &  \plot{reference_fits/ob-3-5-0.05_exp1_0.15.pdf} \cr\ln
%
exp2, ob-1-20-0.05 & \plot{reference_fits/ob-1-20-0.05_exp2_0.07.pdf} & \plot{reference_fits/ob-1-20-0.05_exp2_0.15.pdf} \cr
exp2, ob-2-10-0.05 & \plot{reference_fits/ob-2-10-0.05_exp2_0.07.pdf} & \plot{reference_fits/ob-2-10-0.05_exp2_0.15.pdf} \cr
exp2, ob-3-5-0.05  &  \plot{reference_fits/ob-3-5-0.05_exp2_0.07.pdf} &  \plot{reference_fits/ob-3-5-0.05_exp2_0.15.pdf} \cr\ln
%
exp3, ob-1-20-0.05 & \plot{reference_fits/ob-1-20-0.05_exp3_0.07.pdf} & \plot{reference_fits/ob-1-20-0.05_exp3_0.15.pdf} \cr
exp3, ob-2-10-0.05 & \plot{reference_fits/ob-2-10-0.05_exp3_0.07.pdf} & \plot{reference_fits/ob-2-10-0.05_exp3_0.15.pdf} \cr
exp3, ob-3-5-0.05  &  \plot{reference_fits/ob-3-5-0.05_exp3_0.07.pdf} &  \plot{reference_fits/ob-3-5-0.05_exp3_0.15.pdf} \cr\ln
}}}


\section[results-checks]{Checks}

\> initial point
\>> tried default, test 1 to 3: no difference in results

\> number of fit steps
\>> tried 3 and 10: no difference in results

\> hadronic phase
\>> testes: constant, Bailly, standard: no difference in results

\> form factors
\>> no difference between Puckett, Arrington, Borkowski

\> hadronic modulus: uncertainty in the high $|t|$ part
\>> scaling factor varied by $10\un{\%}$: no difference in results
\>> different high-$|t|$ parametrisations: no difference in results


%----------------------------------------------------------------------------------------------------

\EndText
\bye

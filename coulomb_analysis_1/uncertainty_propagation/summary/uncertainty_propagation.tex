\input slides.tex
\input utf8-t1

\newpage %-------------------------------------------------------------------------------------------

\def\author{J.~Kašpar}
\def\caption{Uncertainty propagation}
\def\date{22 Nov 2018}

\newpage %-------------------------------------------------------------------------------------------
\title{Method}

\> start with the $\d\si/\d t$ histogram used for the standard fits

\> ``modify'' the histogram by adding randomly generated fluctuations to all bins
\>> stat(istical) and syst(ematic) fluctuations generated according to the same matrix as used in the fit $\ch^2$ expression (respecting the correlations)
\>> relative norm(alisation) fluctuations generated according to Gaussian with $\si = 0.055$

\> ``modification'' performed 100 times with different random seeds

\> each of the ``modified'' histograms fitted with the selected fit approaches
\>> calculate ``deltas'' for each parameter ($\rh$, $\si_{\rm tot}$): ``modified'' fit - ``start'' fit
\>> discard fit if $\De\rh > 0.05$ or $\De\si_{\rm tot} > 10\un{mb}$ -- we would hardly accept such results

\> ``delta'' distributions and their RMS summarised on the next slide



\newpage %-------------------------------------------------------------------------------------------
\title{Plots}

\> red: only normalisation fluctuation simulated in the ``modification'' step
\> blue: all kinds of fluctuations simulated in the ``modification'' step

\> RMS values in the legend

\centerline{\fig[16cm]{../plots/hist_cmp.pdf}}



\newpage %-------------------------------------------------------------------------------------------
\title{Validations}

\> validation 1
\>> the procedure from slide 1 was executed with ``$0\un{\si}$'' fluctuations in the ``modification'' step
\>> the fits reproduce the standard results

\> validation 2
\>> for approach 1, the normalisation uncertainty should drive the errors
\>>> correct, on slide 2 the RMS of the red histogram is almost the same as the blue one
\>> for approach 3, in contrary, the normalisation should play negligible role -- the fit should determine it on its own
\>>> correct, on slide 2 the RMS of the red histogram is extremely small

\> validation 3 -- comparison with alternative uncertainty estimates
\>> approach 1: in the preprint we quote $\rh$ uncertainty of $0.01$
\>>> compatible with the value from slide 2
\>> approach 2: rough estimate made for the September LHCC
\>>> compatible with the values from slide 2
\>> approach 3, step f: the fit presented yesterday (21 Nov 2018) yields $\rh$ and $\si_{\rm tot}$ uncertainties of $0.0103$ and $3.27\un{mb}$, resp.
\>>> compatible with the values from slide 2


\newpage %-------------------------------------------------------------------------------------------
\title{Updated result table}

(many digits for our internal use)

\cBlack
\centerline{\tab{\bln
										& \rh 			& \si_{\rm tot}\ung{mb}	\cr\bln
\hbox{CERN-EP-2017-321}					& -				& 110.6 \pm 3.4			\cr\ln
\hbox{CERN-EP-2017-335}					& 0.09 \pm 0.01 & -						\cr\bln
%
\hbox{approach 1: norm.~fixed}			& 0.08795 \pm 0.0081	& 111.777 \pm 3.158	\cr\ln
\hbox{approach 2: norm.~constrained}	& 0.08597 \pm 0.0075	& 111.347 \pm 3.175	\cr\ln
\hbox{approach 3: norm.~free, step d}	& 0.08478 \pm 0.0104	& 110.318 \pm 3.954 \cr\ln
\hbox{approach 3: norm.~free, step f}	& 0.09906 \pm 0.0092	& 109.349 \pm 3.005	\cr\bln
}
}

\> weighted average of CERN-EP-2017-321 and approach 3, step d:
\>> the preprint datum taken with 1 decimal digit as above

\cThird
$$\si_{\rm tot} = (110.480 \pm 2.578)\un{mb}$$


\vfil
\eject
\bye
